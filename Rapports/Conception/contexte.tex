\chapter{Contexte}

Ce projet a pour but de fournir un programme permettant de concevoir des bases d'apprentissage automatiquement pour l'entraînement de systèmes de reconnaissance d'écriture manuscrite. Il sera notamment exploité par les \href{http://archives.ille-et-vilaine.fr/fr}{archives d'Illes-et-Vilaine} ainsi que la startup \href{http://www.doptim.eu}{Doptim}. Les reconnaisseurs utilisés pouvant être multiples, il faut que ce projet puisse facilement évoluer, qu'une partie du projet puisse être remplacé par un morceau plus adapté au reconnaisseur choisi. Ainsi, tous les modules de notre projet et non seulement l'interface avec le reconnaisseur doivent pouvoir être remplacés par l'implémentation choisie par l'utilisateur. Par exemple, nous avons choisi une base de données intégée avec \textit{sqlite} mais celle-ci ne peut gérer facilement l'accès concurrentiel à la base de données, ou gérer efficacement une grande quantité de données. Ainsi, l'utilisateur pourrait choisir d'utiliser une autre base de données comme \textit{MySQL} ou \textit{MongoDB}. Nous avons donc du prendre en compte dans l'architecture l'aspect interchangeable de nos modules.