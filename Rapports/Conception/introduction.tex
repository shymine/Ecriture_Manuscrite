\chapter{Introduction}
\pagenumbering{arabic}
\pagestyle{fancy}
\fancyhf{}

Ce projet nous a été proposé par l’équipe \href{https://www-intuidoc.irisa.fr/}{IntuiDoc} de l’\href{https://www.irisa.fr/}{IRISA}, en collaboration avec la startup \href{http://www.doptim.eu}{Doptim} et avec le soutien de Jean-Yves LE CLERC, conservateur du patrimoine aux \href{http://archives.ille-et-vilaine.fr/fr}{archives départementales} d'Ille-et-Vilaine. Tout au long de l’année, nous serons encadrés par Bertrand COÜASNON, enseignant-chercheur membre d'IntuiDoc, Erwan FOUCHÉ, chef de projet chez \href{https://www.soprasteria.com/fr}{Sopra Steria} et Julien BOUVET, ingénieur chez Sopra Steria également. Nous serons aussi accompagnés par Sophie TARDIVEL, responsable et \textit{data scientist} chez Doptim.

\paragraph{}
Ce rapport définira la structure de notre projet, ainsi que les moyens de conception mis en oeuvre pour répondre aux spécifications définies dans les précédents rapports. Certains choix de notre conception sont justifiés par le contexte du projet, que nous rappellerons au besoin. Le chapitre 2 du rapport présentera l'architecture logicielle générale de l'application, ses différentes parties, ainsi que leurs interactions, tandis que les parties suivantes présenteront ces parties de l'architecture plus en détail. La solution technique du projet étant basée sur un modèle client / serveur, nous présenterons le côté client dans le chapitre 3 et le côté serveur dans le chapitre 4. La partie présentant le côté client traitera de l'organisation de l'IHM, tandis que celle qui présente le côté serveur définira sa structure, tout d'abord d'un point de vue général, ce serveur comportant beaucoup d'éléments plus difficiles à visualiser graphiquement, puis pour chaque module individuellement. Nous expliquerons également les interactions entre ces modules. Enfin, nous décrirons le fonctionnement du projet à l'aide de diagrammes de séquence avant de conclure ce rapport. Nous avons décidé de conduire ce projet sur deux itérations. Nous aborderons donc pour chaque partie ce qui est intégré à la première itération, et ce qui sera rajouté dans la seconde.