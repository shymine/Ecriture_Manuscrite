\chapter{Introduction}
\pagenumbering{arabic}
\setcounter{page}{1}
\pagestyle{fancy}
\fancyhf{}
\cfoot[\thepage]{\thepage}

Ce projet nous a été proposé par l’équipe \href{https://www-intuidoc.irisa.fr/}{IntuiDoc} de l’\href{https://www.irisa.fr/}{IRISA}, en étroite collaboration avec la startup \href{http://www.doptim.eu}{Doptim} et avec le soutien de Jean-Yves LE CLERC, conservateur du patrimoine aux \href{http://archives.ille-et-vilaine.fr/fr}{archives départementales} d'Ille-et-Vilaine. Tout au long de l’année, nous serons encadrés par Bertrand COÜASNON, enseignant-chercheur membre d'IntuiDoc, Erwan FOUCHÉ, chef de projet chez \href{https://www.soprasteria.com/fr}{Sopra Steria} et Julien BOUVET, ingénieur chez Sopra Steria également. Nous serons aussi accompagnés par Sophie TARDIVEL, responsable et \textit{data scientist} chez Doptim.

\paragraph{}
Ce rapport définira la structure de notre projet ainsi que les moyens de conception mis en oeuvre pour répondre aux spécifications définies dans les précédents rapports.
Dans ce rapport, nous rappellerons le contexte du projet qui justifie certains choix de notre conception. Nous avons décidé de conduire ce projet sur deux itérations. Nous aborderons donc pour chaque partie ce qui est intégré à la première version et à la seconde. Nous décrirons tout d'abord les choix de technologies pour la réalisation de l'IHM, son organisation ainsi que ses interactions avec le serveur. Puis, nous parlerons de la structure du serveur, tout d'abord d'un point de vue général, puis pour chaque module individuellement ainsi que leurs interactions. Enfin, nous décrirons le fonctionnement du projet à l'aide de diagrammes de séquence avant de conclure ce rapport.