\chapter{Introduction}
\pagenumbering{arabic}
\setcounter{page}{1}
\pagestyle{fancy}
\fancyhf{}
\cfoot[\thepage]{\thepage}

Ce projet nous a été proposé par l’équipe \href{https://www-intuidoc.irisa.fr/}{IntuiDoc} de l’\href{https://www.irisa.fr/}{IRISA}, en étroite collaboration avec la startup \href{http://www.doptim.eu}{Doptim} et avec le soutien de Jean-Yves LE CLERC, conservateur du patrimoine aux \href{http://archives.ille-et-vilaine.fr/fr}{archives départementales} d'Ille-et-Vilaine. Tout au long de l’année, nous serons encadrés par Bertrand COÜASNON, enseignant-chercheur membre d'IntuiDoc, Erwan FOUCHÉ, chef de projet chez \href{https://www.soprasteria.com/fr}{Sopra Steria}, Julien BOUVET, ingénieur chez Sopra Steria également. Nous serons aussi accompagnés par Sophie TARDIVEL, responsable et \textit{data scientist} chez Doptim.

\paragraph{}
Dans ce rapport, nous rappellerons le contexte du projet qui justifie certains choix de conception. Nous décrirons ensuite l’architecture logicielle générale de notre projet ainsi que les diagrammes de séquences illustrant le fonctionnement interne de ce dernier. Puis, nous décrirons plus en détail les deux itérations prévues, la première répondant simplement au cahier des charges, la seconde complétant l'ensemble des fonctionnalités décrites dans le rapport de spécification, et donnant plus d'importance à l'ergonomie de la solution logicielle.