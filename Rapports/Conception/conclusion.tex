\chapter{Conclusion}

\paragraph{}
Dans le cadre de notre projet de 4e année à l’INSA, nous avons pour objectif de conduire un projet depuis la rédaction du cahier des charges jusqu’à sa livraison finale en passant par les spécifications, la conception et le développement. Une fois le cahier des charges et les spécifications réalisées, nous avons réalisé ce rapport, détaillant la conception de notre projet.

\paragraph{}
Nous avons découpé ce rapport en deux parties principales : le côté client et le côté serveur, après avoir rappelé le contexte de notre projet de génération de bases d’apprentissage pour la reconnaissance d’écriture.

\paragraph{}
Dans la première partie, nous avons expliqué nos choix de technologies utilisées pour la réalisation de l’interface. Nous avons ensuite détaillé les différents composants de l’IHM pour la première itération puis pour la deuxième, avant de décrire les interactions entre les pages de l’IHM et entre le front-end et le back-end.

\paragraph{}
La deuxième partie était axée sur le côté serveur de l’application. Nous avons tout d’abord décrit l’architecture générale de notre projet. Nous avons ensuite détaillé la conception des blocs de la découpe des images, de la base de données et de l’interface avec le reconnaisseur, en précisant dans chaque partie l’architecture correspondant aux deux itérations successives ainsi que les interactions de chaque bloc avec le reste de l’application.

\paragraph{}
Ce rapport est émaillé de schémas d’architecture, de diagrammes de classes ainsi que de maquettes de l’interface pour la clarté et la lisibilité, afin de permettre une meilleure compréhension à la lecture.

\paragraph{}
Le projet se poursuivra par une phase de développement afin de produire un premier rendu fonctionnel pour le 27 février.
