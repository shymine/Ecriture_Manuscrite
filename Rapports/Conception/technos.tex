\chapter{Technologies utilisées}

Pour ce projet, différents choix de technologies ont du être pris. En effect, notre projet impliquant de la découpe d'image, du stockage de données, et la gestion d'un application client-serveur, il était plus simple d'utiliser des technologies existantes effectuant ces différentes tâches plutôt que tout recoder.
Nous verrons tout d'abord les choix technologiques concernant la partie serveur, puis nous nous pencherons sur ceux correspondant à la partie client.

\section{Serveur}

Le serveur étant un ensemble de modules, chaque module utilise une technologie différente en lien avec sa fonction. Nous aborderons en premier les technologies nécessaires à la préparations des données, puis sur celles liées à leur stockage. Nous verrons ensuite 

\section{Client}

Concernant le client, nous avons opté pour une application Web afin de permettre une évolution vers un possible contexte multiutilisateur dans le cas où notre projet serait deployé sur un serveur distant et d'où les utilisateurs y accèderaient au travers d'une page web. De nombreux framework web existent et nous avons choisi d'utiliser Angular7, soit la dernière version de celui-ci. Nous avions déjà utilisé ce framework durant un projet précédent ce qui ne nous obligeait pas à réapprendre un nouvel outil, mais à consolider les bases que nous avions dessus afin d'avoir au plus vite une version fonctionnelle. De plus, il possède une bonne documentation et de nombreux guides existent.