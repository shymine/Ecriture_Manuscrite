\section{Préparation des données}

Cette partie du projet a pour but de réaliser un premier traitement sur les
données d’entrée. Ces données sont aux formats GEDI et PiFF, présentés dans
le dernier rapport. Nous devons fournir un logiciel qui soit capable de
traiter ces formats (\textbf{PR_FO_1} et \textbf{PR_FO_2}). Nous avons validé
la proposition que nous avions faite, qui était de choisir PiFF comme format
de traitement unique. Nous fournirons donc un convertisseur de GEDI vers PiFF,
et nous garantirons par modélisation logicielle la possibilité pour les futurs
utilisateurs d’écrire d’autres convertisseurs des formats de leur choix vers
PiFF s’ils le souhaitent (\textbf{GEN_EVOL}).

\paragraph{}
En exploitant le fichier PiFF, ce module du projet génèrera les imagettes qui
formeront les exemples. Les coordonnées utilisées pour la découpe d’image seront
calculées grâce au polygone identifiant le paragraphe présent dans le fichier
d’entrée, ainsi que les détecteur de lignes. Cette découpe pourra être réalisée
soit en lignes, soit en paragraphes, ce qui correspond à deux reconnaisseurs
différents à utiliser par la suite. Par ailleurs, les fonctions de manipulation
d’images utiliseront la bibliothèque graphique OpenCV, que nous avions déjà
sélectionnée auparavant. Les deux options seront disponibles sur le logiciel.

\paragraph{}
Nous devrons récupérer la vérité terrain si elle existe dans le fichier d’entrée,
afin de pouvoir l’afficher sur l’IHM et permettre à l’utilisateur de la corriger
ainsi qu’au reconnaisseur de s’entraîner avec, en lui fournissant des exemples
que notre logiciel aura créés en associant les imagettes à leur transcription.