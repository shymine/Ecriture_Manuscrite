\section{Interactions}

Tous les modules présentés dans la partie précédente doivent interagir entre
eux à plusieurs niveaux. Un contrôleur général permet de gérer la communication
et la coopération des différents modules. Une première architecture de celui-ci
est visible sur le schéma XX.

\paragraph{}
\begin{mdframed}[frametitle={Titre}, innerbottommargin=10]
\begin{center}
%\includegraphics[scale=0.6]{fichier.png}
\end{center}
\end{mdframed}

\paragraph{}
Le diagramme reste général et n’entre pas dans les détails de l’implémentation
qui restent encore à définir. Les couleurs correspondant aux couleurs des
parties expliquées sur la Figure 1. Le contrôleur possède une classe
correspondant à la partie préparation des données (en bleu). Celle-ci contient
des méthodes pour générer des exemples à ajouter à la BDD. Une autre classe
(en jaune) permet de communiquer avec la BDD. Elle contient une méthode pour se
connecter à la BDD, et d’autres pour envoyer, recevoir ou modifier des données
stockées. Une dernière classe (partie en vert) permet ensuite de faire le lien
avec le système de reconnaissance d’écriture manuscrite. Elle contient des
méthodes pour lancer l’apprentissage ou l’évaluation du reconnaisseur, ou pour
l’utiliser sur un exemple en mode production. Le contrôleur communique enfin
avec l’IHM pour traiter les demandes de l’utilisateur et lui renvoyer les
images et transcriptions. L’IHM est traitée par un serveur web.