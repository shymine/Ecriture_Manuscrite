\chapter{Annexes}

\section*{Vocabulaire utile}

Le \textbf{\textit{deep learning}} est une méthode d’apprentissage automatique
basée sur un réseau de neurones. Ainsi, un algorithme utilisant le \textit{deep
learning} apprend par lui-même et devient de plus en plus performant au fur et
à mesure qu’il accumule les exemples. Il suffit de lui spécifier les paramètres
du problème qu’il doit résoudre et lui donner des exemples sur lesquels
s’entraîner. 

\paragraph{}
Une \textbf{base d’apprentissage (ou base d'entraînement)} est un ensemble
d’exemples que l’on fournit à un algorithme de \textit{deep learning} afin que
celui-ci puisse apprendre.

\paragraph{}
La \textbf{vérité terrain} est, dans le cadre de notre projet, un ensemble de
documents numériques correspondant à des documents manuscrits. Ces documents
numériques contiennent la transcription des documents manuscrits ainsi que
diverses informations telles que la position des paragraphes dans ces documents
ou encore les numéros de ligne par paragraphe. Cette vérité terrain a été
établie au préalable par des humains et non de manière automatique.

\paragraph{}
Une \textbf{imagette} correspond, dans le cadre de notre projet, à une partie
de texte manuscrit découpée au format image. Elle peut correspondre à une ligne
ou à un paragraphe du document.

\paragraph{}
Une \textbf{retranscription} est, dans le cadre de notre projet, la
transcription tapée d’un texte manuscrit.

\paragraph{}
Les formats \textbf{GEDI} et \textbf{PiFF} sont des formats de description
d’images. Ils contiennent une image et des métadonnées (comme la position des
paragraphes par exemple). Le format GEDI est le format de la base qui nous est
donnée pour nourrir notre base d’apprentissage. Le format PiFF est un format
de description d’images qui tend à être universel.