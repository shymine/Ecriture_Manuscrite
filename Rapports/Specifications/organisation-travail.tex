\chapter{Organisation du travail}

\section{Répartition des tâches}

Comme le projet peut être découpé en plusieurs blocs qui peuvent être
développés de manière presque indépendante, nous nous sommes répartis
par équipe sur chaque bloc. Certains blocs nécessitent
cependant une base provenant d’autres blocs pour faire des tests.
Par exemple, la partie IHM aura besoin d’une base de données pour
vérifier que la recherche d’imagettes dans celle-ci se déroule bien.
Nous allons donc commencer par développer des prototypes simplifiés
pour que tout le monde puisse améliorer sa partie.

\paragraph{}
Voici la répartition que nous avons effectuée sur les différents blocs :

\begin{center}

\begin{tabular}{ | l | l | }
	\hline
	\multicolumn{2}{ | c | }{ \textbf{Bloc 1 : Préparation des données} } \\
	\hline
	\textbf{Spécification} && \textbf{Description} \\
	\hline
	PR_FO_1 & Traiter le format PiFF en interne dans le logiciel \\
	\hline
	PR_FO_2 & Fournir un convertisseur du format GEDI vers PiFF \\
	\hline
	PR_TR_1 & Intégrer une fonction de détection de lignes au logiciel \\
	\hline
	PR_TR_2 & Permettre un découpage des images en lignes \\
	\hline
	PR_TR_3 & Localiser les paragraphes \\
	\hline
	PR_TR_4 & Permettre un découpage des images en paragraphes \\
	\hline
	PR_RE_1 & Associer les images à leur transcription \\
	\hline
	PR_RE_2 & Associer la vérité terrain à une transcription \\
	\hline
	PR_RE_3 & Permettre de générer une vérité terrain si besoin, grâce à un reconnaisseur \\
	\hline
\end{tabular}

\begin{tabular}{ | l | l | }
	\hline
	\multicolumn{2}{ | c | }{ \textbf{Bloc 2 : Stockage des données} } \\
	\hline
	\textbf{Spécification} && \textbf{Description} \\
	\hline
	STO_VER & Stocker des imagettes associées à une vérité terrain \\
	\hline
	STO_USR & Stocker des imagettes associées à une transcription générée par l’utilisateur \\
	\hline
	STO_REC & Stocker des imagettes associées à une transcription générée par un reconnaisseur  \\
	\hline
	STO_SEL & Fournir des méthodes pour accéder aux données stockées  \\
	\hline
	STO_UPD & Fournir des méthodes pour modifier les données stockées \\
	\hline
	STO_INS & Fournir des méthodes pour pouvoir insérer des données à stocker \\
	\hline
	STO_DEL & Fournir des méthodes pour pouvoir supprimer des données stockées \\
	\hline
\end{tabular}

\begin{tabular}{ | l | l | }
	\hline
	\multicolumn{2}{ | c | }{ \textbf{Bloc 3 : Interface avec le reconnaisseur} } \\
	\hline
	\textbf{Spécification} && \textbf{Description} \\
	\hline
	IR_CV & Convertir les données au format d’entrée du reconnaisseur \\
	\hline
	IR_AP & Fournir les données au reconnaisseur \\
	\hline
	IR_EV & Pouvoir lancer une évaluation du reconnaisseur  \\
	\hline
	IR_TR & Pouvoir lancer une transcription d’un document par le reconnaisseur   \\
	\hline
\end{tabular}

\begin{tabular}{ | l | l | }
	\hline
	\multicolumn{2}{ | c | }{ \textbf{Bloc 4 : Interface avec l’utilisateur} } \\
	\hline
	\textbf{Spécification} && \textbf{Description} \\
	\hline
	PEA_GEN_1 & Naviguer entre les différents documents d’un même classeur \\
	\hline
	PEA_GEN_2 & Naviguer entre les différentes pages du manuscrit en cours d’annotation \\
	\hline
	PEA_GEN_3 & Basculer vers la page d’édition des zones de découpe à l’aide d’un bouton \\
	\hline
	PEA_GEN_4 & Posséder une zone montrant la page découpée afin de localiser les zones de découpe du manuscrit, avec la zone en cours d’annotation figurant en une couleur différente \\
	\hline
	PEA_GEN_5 & Afficher la liste des imagettes du manuscrit découpé \\
	\hline
	PEA_GEN_6 & Modifier directement la base de données quand on modifie une vérité terrain \\
	\hline
	PEA_GEN_7 & Supprimer un couple imagette-transcription de la base de données à l’aide d’un bouton prévu à cet effet \\
	\hline
	PEA_GEN_8 & Ajouter à la base de données des documents avec ou sans transcription \\
	\hline
	PEA_GEN_9 & Exporter les données validées afin de lancer un nouvel apprentissage \\
	\hline
	PEA_MA_1 & Placer le curseur sur la première imagette ne possédant pas de transcription à l’ouverture de la page \\
	\hline
	PEA_MA_2 & Un appui sur la touche Entrée permet de positionner le curseur sur l’annotation suivante \\
	\hline
	PEA_MA_3 & Proposer un bouton permettant de basculer vers la prochaine imagette sans vérité-terrain \\
	\hline
	PEA_MCR_1 & Un appui sur Entrée valide les transcriptions zone par zone et passe à la zone suivante \\
	\hline
	PEA_MCR_2 & Modifier une transcription en cliquant dessus pour y positionner le curseur et effectuer manuellement les modifications  \\
	\hline
	PEA_MV_1 & Un appui sur Entrée valide les transcriptions zone par zone \\
	\hline
	PEA_MV_2 & Modifier une transcription en cliquant dessus pour y positionner le curseur et effectuer manuellement les modifications  \\
	\hline
	PDEC_OD_1 & Créer une nouvelle zone à l’aide d’un rectangle \\
	\hline
	PDEC_OD_2 & Les zones ne sont donc pas limitées à des rectangles, mais peuvent prendre la forme d’autres polygones \\
	\hline
	PDEC_OD_3 & Déplacer la zone sélectionnée sur le manuscrit \\
	\hline
	PDEC_OD_4 & Zoomer et dézoomer sur le manuscrit  \\
	\hline
	PDEC_OD_5 & Annuler la dernière action \\
	\hline
	PDEC_OD_6 & Refaire l’action annulée précédemment \\
	\hline
	PDEC_OD_7 & Supprimer toutes les zones de la page pour retourner au manuscrit vierge \\
	\hline
	PDEC_OD_8 & Proposer une option “appliquer la détection de lignes sur la zone” \\
	\hline
	PDEC_OD_9 & Valider la découpe et basculer vers la page d’édition des annotations  \\
	\hline
	PDEC_VM_1 & Posséder un bouton de retour au menu principal \\
	\hline
	PDEC_VM_2 & Permettre à l’utilisateur d’ouvrir un autre manuscrit \\
	\hline
	PDEC_VM_3 & Permettre de naviguer entre les différentes pages du manuscrit \\
	\hline
	PDEC_VM_4 & Permettre à l’utilisateur de se déplacer sur le manuscrit horizontalement et verticalement à l’aide d’un bouton de défilement \\
	\hline
\end{tabular}

\begin{tabular}{ | l | l | }
	\hline
	\multicolumn{2}{ | c | }{ \textbf{Bloc 5 : Général} } \\
	\hline
	\textbf{Spécification} && \textbf{Description} \\
	\hline
	GEN_ERGO & Ergonomie de l‘application \\
	\hline
	GEN_ERGO & Concevoir un logiciel évolutif \\
	\hline
	GEN_ERGO & Fournir un logiciel open source \\
	\hline
\end{tabular}

\end{center}

\paragraph{}

Ainsi nous nous répartissons le travail selon ces blocs.
Enzo Crance et Valentin Fouche s’occupent du bloc 1, Kevin Despoulains et Corentin Guilloux du bloc 2, Timothée Neitthoffer du bloc 3, Laure Du Mesnildot et Charlotte Richard du bloc 4 et Gaël Gendron du bloc 5. 

\paragraph{}
Cette répartition des tâches permet à chaque équipe de spécifier plus en détail
sa partie, et donc de pouvoir estimer plus précisément le temps que prendra le
développement. Nous espérons donc avoir un rapport de planification précis qui
nous permettra d’anticiper d’éventuelles réorganisations d’équipes en fonction
des différentes charges de travail.

\section{Organisation temporelle}

Certains membre du groupe partent en mobilité en Janvier (Kévin DESPOULAINS,
Corentin GUILLOUX, et Gaël GENDRON). Notre premier objectif est donc de
concevoir la base et la documentation de chaque partie avant leur départ.
Nous allons donc anticiper la phase de développement en la débutant dès à
présent. Le diagramme de Gantt précédemment établi se trouve donc modifié de
cette manière :

\paragraph{}
\begin{mdframed}[frametitle={Gantt + légende}, innerbottommargin=10]
\begin{center}
%\includegraphics[scale=0.6]{Usecase_3.png}
\end{center}
\end{mdframed}

\section{Délivrables ?}

\begin{mdframed}[frametitle={Figure N : LOT 1 - X}, innerbottommargin=10]
\begin{center}
\begin{tabular}{ | l | l | }
\hline
{\textbf{Règles}}   &   {\textbf{Description}} \\ \hline
{XX\_1}              &   {Règle XX\_1} \\ \hline
\end{tabular}
\end{center}
\end{mdframed}

Formats
PR\_FO\_1 : Permettre de traiter le format PiFF
PR\_FO\_2 : Permettre de traiter le format GEDI
Traitement
PR\_TR\_1 : Fonction de détection de lignes intégrée au logiciel
PR\_TR\_2 : Permettre un découpage des images en lignes
PR\_TR\_3 : Localiser les paragraphes
PR\_TR\_4 : Permettre un découpage des images en paragraphes
Représentation des données
PR\_RE\_1 : Associer les images à leur transcription
PR\_RE\_2 : Associer la vérité terrain à une transcription
PR\_RE\_3 : Permettre de générer une vérité terrain si besoin, grâce à un reconnaisseur

GEN\_ERGO : Ergonomie du logiciel
GEN\_EVOL : Logiciel évolutif

INT\_FMT : Fournir des données intelligibles par le reconnaisseur
INT\_RES : Récupérer les résultats
GEN\_EVOL : Changer de reconnaisseur (pouvoir installer n’importe quel reconnaisseur)

INT\_MOD : L’utiliser dans plusieurs modes
-Apprentissage
-Evaluation
-Production
