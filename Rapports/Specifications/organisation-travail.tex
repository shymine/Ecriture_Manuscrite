\section{Organisation du travail}

\begin{mdframed}[frametitle={Figure N : LOT 1 - X}, innerbottommargin=10]
\begin{center}
\begin{tabular}{ | l | l | }
\hline
{\textbf{Règles}}   &   {\textbf{Description}} \\ \hline
{XX\_1}              &   {Règle XX\_1} \\ \hline
\end{tabular}
\end{center}
\end{mdframed}

Formats
PR\_FO\_1 : Permettre de traiter le format PiFF
PR\_FO\_2 : Permettre de traiter le format GEDI
Traitement
PR\_TR\_1 : Fonction de détection de lignes intégrée au logiciel
PR\_TR\_2 : Permettre un découpage des images en lignes
PR\_TR\_3 : Localiser les paragraphes
PR\_TR\_4 : Permettre un découpage des images en paragraphes
Représentation des données
PR\_RE\_1 : Associer les images à leur transcription
PR\_RE\_2 : Associer la vérité terrain à une transcription
PR\_RE\_3 : Permettre de générer une vérité terrain si besoin, grâce à un reconnaisseur

GEN\_ERGO : Ergonomie du logiciel
GEN\_EVOL : Logiciel évolutif

INT\_FMT : Fournir des données intelligibles par le reconnaisseur
INT\_RES : Récupérer les résultats
GEN\_EVOL : Changer de reconnaisseur (pouvoir installer n’importe quel reconnaisseur)

INT\_MOD : L’utiliser dans plusieurs modes
-Apprentissage
-Evaluation
-Production