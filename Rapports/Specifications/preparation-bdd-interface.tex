\section{Préparation des données}

Cette partie du projet a pour but de réaliser un premier traitement sur les
données d’entrée. Ces données sont aux formats GEDI et PiFF, présentés dans
le dernier rapport. Nous devons fournir un logiciel qui soit capable de
traiter ces formats (\textbf{PR\_FO\_1} et \textbf{PR\_FO\_2}). Nous avons validé
la proposition que nous avions faite, qui était de choisir PiFF comme format
de traitement unique. Nous fournirons donc un convertisseur de GEDI vers PiFF,
et nous garantirons par modélisation logicielle la possibilité pour les futurs
utilisateurs d’écrire d’autres convertisseurs des formats de leur choix vers
PiFF s’ils le souhaitent (\textbf{GEN\_EVOL}).

\paragraph{}
En exploitant le fichier PiFF, ce module du projet génèrera les imagettes qui
formeront les exemples. Les coordonnées utilisées pour la découpe d’image seront
calculées grâce au polygone identifiant le paragraphe présent dans le fichier
d’entrée, ainsi que les détecteur de lignes. Cette découpe pourra être réalisée
soit en lignes, soit en paragraphes, ce qui correspond à deux reconnaisseurs
différents à utiliser par la suite. Les deux options seront
disponibles sur le logiciel.

\paragraph{}
Nous devrons récupérer la vérité terrain si elle existe dans le fichier d’entrée,
afin de pouvoir l’afficher sur l’IHM et permettre à l’utilisateur de la corriger
ainsi qu’au reconnaisseur de s’entraîner avec, en lui fournissant des exemples
que notre logiciel aura créés en associant les imagettes à leur transcription.

\section{Stockage des données}

\subsection{Objectifs du système de stockage}

Le principal objectif du système de stockage est de pouvoir stocker les données
traitées par notre logiciel. Ainsi, il doit être possible de pouvoir stocker
des imagettes associées à leur transcription (vérité terrain) (\textbf{STO\_VER}).
On doit aussi pouvoir stocker des imagettes associées à une transcription établie
par l’utilisateur (\textbf{STO\_USR}). On devra également pouvoir stocker les
données obtenues par le système de reconnaissance d’écriture manuscrite qui aura
été entraîné grâce à la base d’apprentissage que notre logiciel lui aura fourni.
Ainsi, il faudra aussi stocker des imagettes associées à leur transcription
générée par le reconnaisseur (\textbf{STO\_REC}).

\paragraph{}
En plus de stocker les données, la partie base de données devra pouvoir
communiquer facilement avec les autres blocs de notre logiciel (IHM et
préparateur de données). Il faudra ainsi fournir une bibliothèque afin 
de pouvoir accéder aux données (\textbf{STO\_SEL}) de la base, les modifier
(\textbf{STO\_UPD}), en ajouter (\textbf{STO\_INS}) ou en supprimer
(\textbf{STO\_DEL}).

\section{Interface avec le reconnaisseur}

Cette partie du projet a pour but de faire le lien avec le système de
reconnaissance d’écriture. Située entre la BDD et le reconnaisseur,
l’interface permet de faire transiter des informations entre les deux, sans
contraintes de format. Elle convertit les données stockées en données
intelligibles pour le reconnaisseur (\textbf{IR\_CV}).

\paragraph{}
Pour tester notre programme, nous avons un reconnaisseur d’écriture
manuscrite nommé
\href{https://github.com/jpuigcerver/Laia/tree/master/egs/iam}{Laia} à notre
disposition. Ce reconnaisseur prend en format d’entrée des fichiers dans un
format spécifique. Nous allons donc concevoir un programme qui transforme les
données stockées dans notre base au format adéquat pour que Laia puisse
apprendre. Notre programme ayant pour vocation à être utilisé par différents
systèmes de reconnaissance, nous allons laisser le code de ce programme
\textit{open source} pour que n’importe qui puisse définir son propre
formateur de données (\textbf{IR\_OS}).

\paragraph{}
Notre logiciel pourra également, via cette interface, lancer
l’apprentissage du système de reconnaissance (\textbf{IR\_AP}), lancer une
évaluation (\textbf{IR\_EV}) et lancer une transcription de document
(\textbf{IR\_TR}).
