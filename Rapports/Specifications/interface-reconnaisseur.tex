\section{Interface avec le reconnaisseur}

Cette partie du projet a pour but de faire le lien avec le système de
reconnaissance d’écriture. Située entre la BDD et le reconnaisseur,
l’interface permet de faire transiter des informations entre les deux, sans
contraintes de format. Elle convertit les données stockées en données
intelligibles pour le reconnaisseur (\textbf{INT\_FMT}).

\paragraph{}
Après discussion avec l'encadrant de projet, nous considérerons que le format
PiFF est connu des reconnaisseurs utilisés, \textit{i.e.} nous n'aurons pas à
écrire de convertisseur de PiFF vers un quelconque format d'entrée d'un
\textit{framework} de \textit{deep learning}. Par ailleurs, ceci permet de
rendre le projet totalement indépendant du \textit{framework}
(\textbf{GEN\_EVOL}). Après avoir transmis les données au reconnaisseur, nous
devrons cependant les récupérer pour traiter les résultats (\textbf{INT\_RES}).

\paragraph{}
Enfin, le logiciel à développer devra fonctionner dans trois modes :
\begin{itemize}
\item un mode Apprentissage ;
\item un mode Évaluation ;
\item un mode Production.
\end{itemize}