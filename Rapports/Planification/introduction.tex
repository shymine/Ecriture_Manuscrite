\chapter{Introduction}
\pagenumbering{arabic}
\setcounter{page}{1}
\pagestyle{fancy}
\fancyhf{}
\cfoot[\thepage]{\thepage}

Dans le cadre de notre projet de 4ème année sur le développement d’un générateur de bases d’apprentissage pour des reconnaisseurs d’écriture manuscrite, nous avons effectué une planification initiale des tâches à réaliser. Ce document est le troisième et dernier rapport de la phase d’analyse du projet, il présente et explique nos choix vis-à-vis de la gestion de projet, de la planification et présente les estimations que nous avons pu faire.

\paragraph{}
Dans l’objectif de vous présenter nos choix, nous rappellerons dans un premier temps le contexte, en expliquant quels sont les acteurs, pourquoi ont-ils fait appel à nous et comment nous comptons répondre leur besoin. Dans un second temps, nous décrirons la gestion que nous comptons employer afin de mener ce projet à bout, au travers des méthodes, de l’organisation ainsi que du mode de pilotage. Dans une troisième partie dédiée à l’estimation, nous expliquerons comment nous avons découpé le projet en tâches afin d’obtenir une estimation précise, que ce soit pour le développement ou pour la partie gestion. Enfin, la dernière partie permettra de mettre en valeur le travail de planification et d’estimation réalisé, en présentant l’analyse effectuée grâce à Microsoft Project.
