
\chapter{L'estimation}

\section{Estimation de la partie analyse}

L’estimation des charges de la partie hors développement se résume en deux grands points : le temps passé en réunion ainsi que le temps passé pour la rédaction des rapports et à la préparation des soutenances. Cette partie n’a pas été la plus complexe à estimer car nous avons pu nous baser sur les temps passés lors des trois premiers mois du projet où nous nous sommes concentrés sur la compréhension du cahier des charges et la planification du projet.

\paragraph{}

Pour les deux grands points, le calcul a été sensiblement le même. Dans un premier temps, pour les réunions, nous nous sommes basés sur le temps hebdomadaire passé en réunion depuis le début du projet, au mois de septembre, que nous avons multiplié à la fois par le nombre de semaines restantes et par le nombre de personnes dans le groupe. Dans un  second temps, nous avons calculé le temps nécessaire à la réalisation des prochains rapports. Dans ce cas, nous avons multiplié le temps passé en moyenne par chaque personne pour l’écriture d’un rapport par le nombre de personnes dans le groupe et par le nombre de rapports qu’il nous reste à rendre. Nous avons pu calculer la charge des rapports de cette façon, car nous avons réparti leurs réalisations de façon homogène entre les différents membres du groupe.

\section{Estimation de la partie développement}

Cette partie a été plus compliquée à estimer car nous avions peu de repères. Pour nous rapprocher au maximum d’une estimation fiable, nous avons estimé le temps de développement de chaque partie en considérant les points suivants : 

\begin{itemize}
\item connaissance des technologies utilisées (donc temps d’apprentissage nécessaire estimé) ;
\item temps estimé de conception ;
\item temps estimé de développement .
\end{itemize}

\subsection{Estimation du temps d'apprentissage}

Les différentes technologies qui seront utilisées ont été définies dans le rapport de spécification. Nous savons donc déjà quelles technologies devront être apprises avant de commencer le développement. Le temps d’apprentissage a été défini grossièrement en nous basant sur notre expérience d’auto-formation sur des langages ou d’autres types de technologie.

\newpage

\subsection{Estimation du temps de conception}

La conception ayant majoritairement été réalisée précédemment, cette partie est presque inexistante. Cependant, il est possible que la conception réalisée soit éloignée des réalités techniques. Nous avons donc prévu un temps de remodélisation si nous nous rendons compte à un moment qu’une modélisation n’est pas la meilleure.

\subsection{Estimation du temps de développement}

Cette partie de l’estimation est la plus compliquée à estimer. En effet, il n’est pas possible de prévoir les problèmes que nous allons rencontrer. C’est pourquoi, nous nous sommes basés sur le temps de développement d’autres projets que nous avons réalisés en y ajoutant une marge au cas où nous serions confrontés à un problème.


