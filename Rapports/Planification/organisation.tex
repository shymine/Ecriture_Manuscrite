\chapter{Organisation : la gestion de projet}

\section{Découpage du projet}

Le projet peut être découpé en quatre blocs distincts. Nous avons donc réparti chacun des membres de l’équipe sur ces blocs. Chaque bloc comprendra une partie de développement, l’écriture de la documentation ainsi que les tests logiciels. Cette façon de découper le projet possède des avantages comme des inconvénients.

\paragraph{}

Le premier avantage est que nous pouvons faire avancer les différentes parties du projet en parallèle et donc de gagner un temps précieux sur sa réalisation globale. Le second avantage est que tous les membres de l’équipe n’ont pas à se former sur la totalité des technologies que nous allons utiliser. Cela permet de gagner encore plus de temps. Le premier inconvénient est que seulement certaines personnes seront compétentes sur un bloc. Par conséquent, lors d’absences des mêmes membres d’un bloc, ce dernier sera gelé. Il faudra aussi penser à prendre cet inconvénient en compte lorsque trois d’entre nous partiront en mobilité. Le second inconvénient est qu’il peut être difficile de regrouper les blocs une fois qu’ils seront développés. Pour pallier ce dernier point, un bloc interface commun à chaque partie servira de liaison entre les différents modules. Ainsi, les modules ne devront s’accorder que sur un bloc plutôt que sur tous les autres.

\paragraph{}

Voici les blocs que nous avons défini :

\begin{itemize}
\item traitement des données en entrées ;
\item création et communication avec la base de données  ;
\item IHM ;
\item interface avec les reconnaisseurs .
\end{itemize}

\section{Cycle de production}

Étant donné le contexte scolaire dans lequel est réalisé ce projet, baser notre gestion de projet sur un cycle en V itératif nous a semblé particulièrement adapté. Ayant des cours à travailler à côté, nous ne pouvons donc pas consacrer l’intégralité de notre temps au projet. Effectuer des méthodes agiles (avec des sprints par exemple) ne semble donc pas adapté, de par les contraintes causées par les autres matières (examens, rendus de rapports et de projets, ...). Par ailleurs, notre
façon de travailler lors de la phase d’analyse nous a conforté dans cette idée. Nous proposons donc de fonctionner en suivant cette méthode pour les livrables jusqu’à la phase de conception, puis en suivant un cycle en V itératif pour la réalisation logicielle. 

\paragraph{}

Nous avons ainsi défini deux cycles ou itérations. Le premier ayant déjà débuté et se terminant fin février, le second débutant à la suite du premier et allant jusqu’à la fin du projet. L’objectif du premier cycle est d’anticiper l’inconvénient d’incompatibilité entre les blocs. Nous allons donc, avant la fin de ce cycle, développer une base de chaque bloc qui devra être capable de communiquer avec les autres blocs. Cette première partie sera conçue d’une traite et sera fonctionnelle et testable par le client.

\newpage

\paragraph{}

Si le logiciel ne comporte pas d’erreurs et qu’il fonctionne correctement (la communication entre les blocs se fait sans problèmes), nous pourrons nous lancer dans le développement approfondi de nos parties, de manière plus itérative, c’est-à-dire ajouter progressivement des fonctionnalités en s’assurant que le logiciel fonctionne toujours normalement. Dans le cas contraire, il faudra d’abord résoudre les problèmes de compatibilité.

\section{Mode de pilotage}

Afin de gérer la liste des tâches, Trello nous permettra de les attribuer aux différents développeurs selon la planification prévue et de suivre en temps réel leur avancement. Microsoft Project nous permettra de calculer le temps passé sur chacune d’entre elles, afin de comparer notre avancement réel avec les prévisions. Comme nous avons déjà effectué la planification, nous serons en mesure de comparer notre avancement réel face à celui supposé, afin de déceler d’éventuels retards, et d’y apporter des solutions comme, par exemple, laisser une fonctionnalité moins importante de côté pour se focaliser sur l’essentiel ou alors demander un allongement de délai pour la réalisation. Ces risques ainsi que les actions à réaliser en réponse ont été décrits précédemment dans ce rapport.

\paragraph{}

Nous estimons que deux cycles sont suffisants car, une fois les problèmes de compatibilité entre les blocs résolus, les améliorations apportées à chaque blocs seront internes à ceux-ci et donc n’induiront que très peu (si ce n’est aucun) problème de compatibilité supplémentaire. Faire plus de cycles n’est donc pas nécessaire mais pourra être envisagé si le besoin se fait ressentir au cours du second semestre. Ce mode de pilotage nous permettra ainsi d’avoir un logiciel fonctionnel au 9 Mai 2019, date de livraison du projet, utilisable par le client et pouvant être amélioré dans le futur.















