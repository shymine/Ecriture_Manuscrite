\chapter{Conclusion}

Ce rapport et les deux présentations qui donnent suite à ce dernier représentent la fin
de la phase d’analyse et de spécifications de notre projet de quatrième année.

\paragraph{}

Tout au long de la phase de planification, nous avons vu et estimé la durée de chacune des tâches du projet. Cela nous a permis d’établir une planification initiale des tâches, du début de la phase de conception jusqu’au rendu de la dernière version du projet, en y incluant la gestion de projet. Nous avons ainsi découpé le projet en 6 tâches principales, elles-mêmes découpées en une multitude de sous-tâches détaillées sur le diagramme de Gantt de la partie précédente. En ayant considéré 1 heure de travail par personne et par jour, ainsi qu’une semaine dédiée   au projet contenant 7 heures de travail par persone et par jour,  notre première estimation annonce la fin du projet au 26 avril 2018. Évidemment, nous ne sommes pas à l’abri d’imprévus, c’est pourquoi nous avons quelques jours de travail disponibles avant la fin du projet.

\paragraph{}

Cette phase, dont l’importance ne nous semblait, à priori, pas capitale, nous a finalement paru très importante : elle nous a permis de voir que la gestion de projet prenait une grande place dans la réalisation dudit projet, mais également que ce dernier n’est pas trop ambitieux par rapport à nos capacités. La majeure difficulté de la phase de planification a été d’estimer correctement la durée des tâches, chacun de nous ayant souvent une estimation assez différente. Pour palier ce problème, nous avons décidé de faire la moyenne de toutes nos estimations afin d’obtenir un résultat certainement plus proche de la réalité que nos estimations initiales.

\paragraph{}

Cette phase de planification nous a beaucoup apporté, nous nous sommes rendus compte de la difficulté de la répartition des charges ainsi que de la difficulté d’estimer la durée d’une tâche. Par ailleurs, cela nous a également permis de nous rendre compte de l’importance de bien découper toutes ces tâches en sous-tâches, afin de ne rien oublier et ainsi pouvoir terminer le projet dans les délais annoncés.
