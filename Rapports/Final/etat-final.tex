\chapter{État de finalisation du projet}

\paragraph{}
En cette fin de projet, nous sommes en capacité de fournir une application fonctionnelle qui remplit les exigences principales du cahier des charges.

Côté back

\paragraph{Front-end}
Notre IHM contient trois pages : une page d'accueil, une page d'annotation manuelle et une page de validation des transcriptions.
\newline
La page d'accueil permet de naviguer dans les projets et les documents afin de choisir le manuscrit à annoter. Elle propose également de créer de nouveaux projets ou de supprimer des documents ou des projets entiers.
\newline
Sur la page d'annotation manuelle, l'utilisateur peut visualiser le manuscrit découpé ligne par ligne ainsi que la transcription associée sous chaque imagette. Il peut modifier cette transcription en la tapant au clavier. Si un exemple (un couple imagette-transcription) semble manquer de pertinence, l'utilisateur peut le cacher à l'aide d'une croix cliquable située dans le coin supérieur droit de l'imagette.
\newline
Enfin, la page de validation permet une relecture rapide des transcriptions pour effectuer une validation finale.

\paragraph{}
Nous avons développé l'ergonomie de notre application autant que possible dans le temps restreint dont nous disposions et nous avons tenté de fournir une application aussi intuitive que possible en facilitant son utilisation et en simplifiant les gestes requis par l'utilisateur (minimum de mouvements de souris, raccourcis clavier...).



décrire de façon assez exhaustive à quoi ressemble l'appli maintenant, ce qui a été fait, lister les fonctionnalités principales