\chapter{État de finalisation du projet}

\paragraph{}
En cette fin de projet, nous sommes en capacité de fournir une application fonctionnelle qui remplit les exigences principales du cahier des charges.

\subsection{Back-end}

\subsection{Front-end}
\paragraph{}
Notre IHM contient trois pages : une page d'accueil, une page d'annotation manuelle et une page de validation des transcriptions.
\paragraph{}
La page d'accueil permet de naviguer dans les projets et les documents afin de choisir le manuscrit à annoter. Elle propose également de créer de nouveaux projets en choisissant le nom du projet, le reconnaisseur associé et les documents qui le composent. L'utilisateur peut également supprimer des documents faisant partie d'un projet ou encore des projets entiers.
\paragraph{}
Sur la page d'annotation manuelle, l'utilisateur peut visualiser le manuscrit découpé ligne par ligne ainsi que la transcription associée sous chaque imagette. Il peut modifier cette transcription en la tapant au clavier. Si un exemple (un couple imagette-transcription) semble manquer de pertinence, l'utilisateur peut le cacher à l'aide d'une croix cliquable située dans le coin supérieur droit de l'imagette.
\paragraph{}
Enfin, la page de validation permet une relecture rapide des transcriptions pour effectuer une validation finale. L'utilisateur peut également cacher un exemple qui manque de pertinence, comme sur la page d'annotation. Un simple appui sur \texttt{Entrée} permet de valider d'un seul coup toutes les transcriptions affichées sur la page courante.

\paragraph{}
Nous avons développé l'ergonomie de notre application autant que possible dans le temps restreint dont nous disposions et nous avons tenté de fournir une application aussi intuitive que possible en facilitant son utilisation et en simplifiant les gestes requis par l'utilisateur (minimum de mouvements de souris, raccourcis clavier...).
