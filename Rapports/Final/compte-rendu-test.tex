\chapter{Compte rendu des phases de test}

Nous avons réalisé des tests successifs pour chaque composant de l'application : la base de données, la découpe d'images, l'API REST, et l'interface. Tous ces tests nous ont permis de nous rendre compte de certaines erreurs qui avaient été commises et de les corriger. Les tests ont été réalisés au fur et à mesure de l'implémentation pour éviter d'accumuler des erreurs que nous n'aurions pas détectées aussi facilement en testant le logiciel une fois totalement construit d'un bout à l'autre.

\section{Base de données}
En ce qui concerne la base de données, nous avons utilisé des tests unitaires en Scala (notamment grâce à la bibliothèque \href{http://www.scalatest.org/}{ScalaTest}) afin de tester les fonctionnalités de la base. Nous avons testé toutes les différentes opérations possibles, qui permettent de stocker des données de types différents, de les modifier, et de les supprimer.

\section{Découpe d'images}
Pour le traitement d'images, le problème est qu'aucun test non humain ne permet de déterminer si la découpe a été bien réalisée. Nous n'avons donc pas eu d'autre choix que de vérifier manuellement les images obtenues pour vérifier le bon fonctionnement de la découpe. Nous avons sélectionné au hasard un certain nombre d'images parmi celles qui ont été produites pour vérifier le découpage. Divers scripts ont dû être réalisés pour appuyer l'application serveur. Chaque script a été testé lors de son développement, pour s'assurer qu'il fonctionnait correctement.

\section{API REST}
Afin de tester l'API REST, nous avons utilisé \href{https://www.getpostman.com/}{Postman}, un logiciel particulier dédié au test d'API à partir duquel nous avons pu vérifier le bon fonctionnement de l'envoi de nos requêtes ainsi que les réponses obtenues.

\section{IHM}
Enfin, pour tester l'interface web, nous avons effectué une série de tests utilisateur en essayant chaque bouton et chaque fonctionnalité manuellement, et en vérifiant les résultats à l'aide de la console intégrée au navigateur web.