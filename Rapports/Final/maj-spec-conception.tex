\chapter{Mise à jour des rapports de spécification et conception}

\paragraph{}
Nous avons déterminé les spécifications en novembre et la conception en février. Au fur et à mesure de l'évolution du projet pendant le second semestre, nous y avons apporté des mises à jour que nous allons détailler ci-dessous. Nous avons repris la liste des spécifications du rapport de conception, les lignes en bleu sont les règles que nous avons rajoutées et les lignes en rouge sont celles que nous avons dû écarter par manque de temps.

%\rowcolor{red}
\begin{center}

    \begin{tabular}{ | l | l | }
        \hline
        \multicolumn{2}{ | c | }{ \textbf{Bloc 1 : Préparation des données} } \\
        \hline
        \textbf{Spécification} & \textbf{Description} \\
        \hline
        PR\_FO\_1 & Traiter le format PiFF en interne dans le logiciel \\
        \hline
        PR\_FO\_2 & Fournir un convertisseur du format GEDI vers PiFF \\
        \hline
        PR\_TR\_1 & Intégrer une fonction de détection de lignes au logiciel \\
        \hline
        PR\_TR\_2 & Permettre un découpage des images en lignes \\
        \hline
        PR\_TR\_3 & Localiser les paragraphes \\
        \hline
        PR\_TR\_4 & Permettre un découpage des images en paragraphes \\
        \hline
        PR\_RE\_1 & Associer les images à leur transcription \\
        \hline
        PR\_RE\_2 & Associer la vérité terrain à une transcription \\
        \hline
        \rowcolor{red}PR\_RE\_3 & Permettre de générer une vérité terrain si besoin, grâce à un reconnaisseur \\
        \hline
    \end{tabular}

    \paragraph{}
    \begin{tabular}{ | l | l | }
        \hline
        \multicolumn{2}{ | c | }{ \textbf{Bloc 2 : Stockage des données} } \\
        \hline
        \textbf{Spécification} & \textbf{Description} \\
        \hline
        STO\_VER & Stocker des imagettes associées à une vérité terrain \\
        \hline
        STO\_USR & Stocker des imagettes associées à une transcription générée par l’utilisateur \\
        \hline
        STO\_REC & Stocker des imagettes associées à une transcription générée par un reconnaisseur  \\
        \hline
        STO\_SEL & Fournir des méthodes pour accéder aux données stockées  \\
        \hline
        STO\_UPD & Fournir des méthodes pour modifier les données stockées \\
        \hline
        STO\_INS & Fournir des méthodes pour pouvoir insérer des données à stocker \\
        \hline
        STO\_DEL & Fournir des méthodes pour pouvoir supprimer des données stockées \\
        \hline
    \end{tabular}

    \paragraph{}
    \begin{tabular}{ | l | l | }
        \hline
        \multicolumn{2}{ | c | }{ \textbf{Bloc 3 : Interface avec le reconnaisseur} } \\
        \hline
        \textbf{Spécification} & \textbf{Description} \\
        \hline
        IR\_CV & Convertir les données au format d’entrée du reconnaisseur \\
        \hline
        IR\_AP & Fournir les données au reconnaisseur \\
        \hline
        \rowcolor{red}IR\_EV & Pouvoir lancer une évaluation du reconnaisseur  \\
        \hline
        \rowcolor{red}IR\_TR & Pouvoir lancer une transcription d’un document par le reconnaisseur   \\
        \hline
    \end{tabular}

    \begin{tabular}{ | l | p{0.8\linewidth} | }
        \hline
        \multicolumn{2}{ | c | }{ \textbf{Bloc 4 : Interface avec l’utilisateur} } \\
        \hline
        \textbf{Spécification} & \textbf{Description} \\
        \hline
        PEA\_GEN\_1 & Valider un ensemble d'annotations \\
        \hline
        PEA\_GEN\_2 & Éditer manuellement les transcriptions \\
        \hline
        PEA\_GEN\_3 & Corriger les annotations proposées par un reconnaisseur externe à l'application \\
        \hline
        PEA\_GEN\_4 & Envoyer les modifications à la base de données lorsque la vérité-terrain d’une imagette est modifiée \\
        \hline
        PEA\_GEN\_5 & Ignorer un couple imagette-transcription s’il n’est pas pertinent \\
        \hline
        PEA\_GEN\_6 & Regrouper les documents en projets \\
        \hline
        PEA\_GEN\_7 & Sélectionner d’abord le projet puis le document sur lequel l’utilisateur veut travailler à l’ouverture de l’application \\
        \hline
        PEA\_GEN\_8 & Créer un nouveau projet \\
        \hline
        \rowcolor{red}PEA\_GEN\_9 & Basculer vers la page de découpe des zones \\
        \hline
        PEA\_GEN\_10 & Basculer vers la page d’édition des annotations \\
        \hline
        PEA\_GEN\_11 & Basculer vers la page de validation des transcriptions \\
        \hline
        \rowcolor{blue}PEA\_GEN\_12 & Supprimer un projet ou un document d'un projet \\
        \hline
        \rowcolor{red}PDEC\_OD\_1 & Créer une nouvelle zone à l’aide d’un rectangle (outil "nouvelle sélection") \\
        \hline
        \rowcolor{red}PDEC\_OD\_2 & Pouvoir modifier la position des sommets des rectangles \\
        \hline
        \rowcolor{red}PDEC\_OD\_3 & Rajouter des sommets à la zone \\
        \hline
        \rowcolor{red}PDEC\_OD\_4 & Changer le type de la zone avec un menu déroulant \\
        \hline
        \rowcolor{red}PDEC\_OD\_5 & Déplace la zone sélectionnée sur le document (outil “déplacer”) \\
        \hline
        \rowcolor{red}PDEC\_OD\_6 & Zoomer et dézoomer sur le document (outils “zoom +” et “zoom -”) \\
        \hline
        \rowcolor{red}PDEC\_OD\_7 & Annuler la dernière action (outil “annuler”) \\
        \hline
        \rowcolor{red}PDEC\_OD\_8 & Refaire l’action annulée précédemment (outil “refaire”) \\
        \hline
        \rowcolor{red}PDEC\_OD\_9 & Supprime toutes les zones de la page pour retourner au document vierge (outil “réinitialiser”) \\
        \hline
        \rowcolor{red}PDEC\_OD\_10 & Applique un détecteur de lignes sur la zone sélectionnée (outil “appliquer la détection de lignes sur la zone”) \\
        \hline
        \rowcolor{red}PDEC\_OD\_11 & Continuer la découpe du document sur la page suivante \\
        \hline
        \rowcolor{red}PDEC\_OD\_12 & Passer à l’édition des annotations sur la page qu’il vient de découper \\
        \hline
        \rowcolor{red}PDEC\_OD\_13 & Exporter la page découpée au format PiFF afin de soumettre les données à un reconnaisseur externe à l’application \\
        \hline
        \rowcolor{red}PDEC\_OD\_14 & Posséder un bouton de retour au menu principal \\
        \hline
        \rowcolor{red}PDEC\_OD\_15 & Permettre à l’utilisateur de se déplacer sur le manuscrit à l’aide d’un scroll horizontal et vertical \\
        \hline
        PEMA\_1 & Placer le curseur sur la première imagette ne possédant pas de transcription \\
        \hline
        PEMA\_2 & Positionner le curseur sur l’annotation suivante en appuyant sur Entrée \\
        \hline
        PEMA\_3 & Proposer un raccourci clavier permettant de basculer vers la prochaine imagette sans vérité-terrain \\
        \hline
        \rowcolor{red}PEMA\_4 & Posséder un bouton intitulé “modifier les zones du manuscrit” \\
        \hline
        PEMA\_5 & Afficher la liste des imagettes du document découpé \\
        \hline
        PEMA\_6 & Ignorer un couple imagette-transcription s’il n’est pas pertinent \\
        \hline
        PEMA\_7 & Basculer vers la page de validation des transcriptions \\
        \hline
        PCORIA\_1 & Valider les transcriptions zone par zone et passer à la zone suivante avec un simple appui sur Entrée \\
        \hline
        PCORIA\_2 & Pouvoir modifier une annotation fausse en cliquant dessus pour y positionner son curseur et en effectuant ses modifications manuellement \\
        \hline
        PCORIA\_3 & La zone de visualisation des imagettes se présente de la même manière que sur la page d’édition manuelle des transcriptions \\
        \hline
        PCORIA\_4 & Posséder également la fonctionnalité de mise à l’écart d’un couple imagette-transcription \\
        \hline
        PCORIA\_5 & Basculer vers la page de validation des transcriptions \\
        \hline
    \end{tabular}

    \begin{tabular}{ | l | p{0.8\linewidth} | }
        \hline
        \textbf{Spécification} & \textbf{Description} \\
        \hline
        PVAL\_1 & Accéder à la page de validation depuis le menu principal \\
        \hline
        PVAL\_2 & Accéder à cette page de validation depuis les pages d’édition manuelle des annotations et de correction des transcriptions proposées par le reconnaisseur \\
        \hline
        PVAL\_3 & Valider les transcriptions zone par zone et passer à la zone suivante avec un simple appui sur Entrée \\
        \hline
        PVAL\_4 & Pouvoir modifier une annotation fausse en cliquant dessus pour y positionner son curseur et en effectuant ses modifications manuellement \\
        \hline
        PVAL\_5 & Indiquer si les transcriptions ont été fournies manuellement par un humain ou si elles proviennent d’un reconnaisseur \\
        \hline
        PVAL\_6 & Faire figurer une fenêtre montrant la page entière découpée en zones avec la zone courante dans une couleur différente \\
        \hline
        PVAL\_7 & Valider le travail pour de bon et fermer le document à l’aide d’un bouton prévu à cet effet \\
        \hline
    \end{tabular}

    \paragraph{}
    \begin{tabular}{ | l | p{0.7\linewidth} | }
        \hline
        \multicolumn{2}{ | c | }{ \textbf{Bloc 5 : Lien entre les blocs précédents} } \\
        \hline
        \textbf{Spécification} & \textbf{Description} \\
        \hline
        LINK\_PR\_STO & Envoyer les données en entrée vers le système de stockage \\
        \hline
        LINK\_STO\_IHM & Extraire les données pour les fournir à l’IHM \\
        \hline
        LINK\_STO\_IR & Extraire les données pour les fournir au système de reconnaissance \\
        \hline
        LINK\_IHM\_STO & Envoyer les demandes de l’IHM au système de stockage \\
        \hline
        LINK\_IHM\_IR & Envoyer les résultats du reconnaisseur vers le système de stockage \\
        \hline
        LINK\_COH & Fournir un logiciel composés de blocs communiquant entre eux de manière fonctionnelle et cohérente \\
        \hline
    \end{tabular}

    \paragraph{}
    \begin{tabular}{ | l | l | }
        \hline
        \multicolumn{2}{ | c | }{ \textbf{Bloc 6 : Général} } \\
        \hline
        \textbf{Spécification} & \textbf{Description} \\
        \hline
        GEN\_ERGO & Ergonomie de l‘application \\
        \hline
        GEN\_ERGO & Concevoir un logiciel évolutif \\
        \hline
        GEN\_ERGO & Fournir un logiciel open source \\
        \hline
    \end{tabular}

\end{center}