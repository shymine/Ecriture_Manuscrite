\chapter{Contexte du projet}

Comme énoncé dans l’introduction, ce projet est en collaboration avec l’entreprise
Doptim et les archives départementales d’Ille-et-Vilaine. Le produit de notre travail
leur sera donc mis à disposition afin qu’ils puissent l’exploiter.

\section{Les archives départementales d’Ille-et-Vilaine}

Les archives départementales d’Ille-et-Vilaine, situées à Rennes, regroupent plusieurs
millions de documents manuscrits datant, pour les plus anciens, du début du XI\up{ème} siècle.
Ces documents n’y sont pas seulement entreposés mais aussi numérisés pour que n’importe qui
puisse les consulter, soit en version physique aux archives, soit en version numérique
depuis n’importe où. Il existe actuellement des moteurs de recherche permettant de retrouver
les documents grâce à des annotations ou des mots-clés, mais leur utilisation reste limitée
car les documents sont souvent peu lisibles. L'outil que notre projet va apporter aux
archivistes leur permettra de retranscrire la totalité du contenu d’un document manuscrit
sous forme de texte numérique pour que la recherche soit plus efficace et la consultation
plus agréable. Cet outil pourra facilement être intégré à la chaîne d’archivage des documents
puisqu’il pourra être exécuté juste après la numérisation, sans grande intervention humaine.

\section{Doptim}

Doptim est une entreprise spécialisée dans le \textit{Big Data} et l’analyse de données.
Elle a été fondée par Sophie TARDIVEL, qui sera notre contact dans l’entreprise.
Son but premier est de créer une communauté de \textit{data scientists} et de
\textit{data engineers} qui auraient l’ambition d’optimiser et de maîtriser la gestion
des données. Doptim est aujourd’hui investie dans un projet de service en ligne permettant
aux généalogistes de gagner du temps dans la fouille et le décryptage de documents numériques.
Notre projet pourrait donc être ajouté au leur pour simplifier la lecture de documents anciens.

\paragraph{}
Bien sûr, l’outil final de notre projet qui a pour but d’être compatible avec plusieurs
types de reconnaisseurs d’écriture manuscrite ne sera pas notre propriété. Nous le rendrons
public et utilisable par qui le veut. La collaboration que nous avons avec les archives
départementales d’Ille-et-Vilaine et Doptim nous permettent d’orienter notre projet afin de
nous faire une idée de ce qu’un utilisateur pourrait en attendre (notamment au niveau de l’IHM).
Ces deux collaborations ne sont pas distinctes puisque Doptim et les archives travaillent déjà ensemble.