\chapter{Contexte du projet}

Comme énoncé dans l’introduction, ce projet est en collaboration avec l’équipe Intuidoc de l’IRISA,
l’entreprise Doptim et les archives départementales d'Ille-et-Vilaine. Le produit de notre travail
leur sera donc mis à disposition, directement ou indirectement, afin qu’ils puissent l’exploiter.

\section{L’équipe de recherche IntuiDoc}

Dans le cadre de ses recherches, l’équipe IntuiDoc de l’IRISA cherche à faire avancer le domaine
de la reconnaissance d’écriture manuscrite afin de rendre plus accessibles des textes anciens qui
sont souvent peu compréhensibles. Il n’est pas simple d’écrire un programme qui reconnaît les textes
manuscrits, c’est pourquoi la plupart des systèmes de reconnaissance d’écriture sont basés sur des
algorithmes intelligents. Ces algorithmes sont souvent formés de réseaux de neurones qui ont besoin
d’apprendre à reconnaître les différents caractères, quels que soient la langue et le style du rédacteur.
Pour apprendre, ils ont besoin d’un grand nombre d’exemples (plusieurs milliers) qui sont longs
à construire à la main. 

\paragraph{}
Dans ce contexte, les exemples d’apprentissage (appelés base d’apprentissage dans la suite de ce rapport)
sont des associations entre les textes manuscrits et leurs retranscriptions informatiques. Ainsi,
l’algorithme apprend à reconnaître les caractères en comparant sa sortie avec la retranscription fournie.
Notre projet, qui a été proposé par l’équipe IntuiDoc, est donc de construire un système qui
permet de générer des bases d’apprentissage à partir de données brutes (mais dans des formats précis)
de manière automatique.

\section{Doptim}

Doptim est une entreprise spécialisée dans le \textit{Big Data} et l’analyse de données.
Elle a été fondée par Sophie TARDIVEL, qui sera notre contact dans l’entreprise.
Son but premier est de créer une communauté de \textit{data scientists} et de
\textit{data engineers} qui auraient l’ambition d’optimiser et de maîtriser la gestion
des données. Doptim est aujourd’hui investie dans un projet de service en ligne permettant
aux généalogistes de gagner du temps dans la fouille et le décryptage de documents numériques.
De plus, Doptim est en collaboration avec les archives départementales d’Ille-et-Vilaine.
Cette collaboration a pour but de créer un système de reconnaissance d’écriture manuscrite qui
aidera les archivistes. Notre projet pourrait donc être utilisé au sein du leur pour qu’ils
puissent générer plus rapidement leurs bases d’apprentissage.

\section{Les archives départementales d’Ille-et-Vilaine}

Les archives départementales d’Ille-et-Vilaine, situées à Rennes, regroupent plusieurs
millions de documents manuscrits datant, pour les plus anciens, du début du XI\up{ème} siècle.
Ces documents n’y sont pas seulement entreposés mais aussi numérisés pour que n’importe qui
puisse les consulter, soit en version physique aux archives, soit en version numérique
depuis n’importe où. Il existe actuellement des moteurs de recherche permettant de retrouver
les documents grâce à des annotations ou des mots-clés, mais leur utilisation reste limitée
car les documents sont souvent peu lisibles. Leur collaboration avec Doptim a pour but de
retranscrire la totalité du contenu d’un document manuscrit sous forme de texte numérique
pour que la recherche soit plus efficace et la consultation plus agréable. Cet outil pourra
facilement être intégré à la chaîne d’archivage des documents puisqu’il pourra être exécuté
juste après la numérisation, sans intervention humaine.

\paragraph{}
L’outil final de notre projet qui a pour but d’être compatible avec plusieurs types de
reconnaisseurs d’écriture manuscrite ne sera pas notre propriété. Nous le rendrons public et
utilisable par tout le monde. La collaboration que nous avons avec l’IRISA, les archives
départementales d’Ille-et-Vilaine et Doptim nous permettent d’orienter notre projet afin de
nous faire une idée de ce qu’un utilisateur pourrait en attendre.