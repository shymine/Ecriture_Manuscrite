\chapter{Introduction}
\pagenumbering{arabic}
\setcounter{page}{1}

Ce projet nous a été proposé par l’équipe \href{https://www-intuidoc.irisa.fr/}{IntuiDoc}
de l’\href{https://www.irisa.fr/}{IRISA}, en étroite collaboration avec la startup
\href{http://www.doptim.eu}{Doptim} et avec le soutien de Jean-Yves LE CLERC, conservateur du
patrimoine aux \href{http://archives.ille-et-vilaine.fr/fr}{archives départementales} d'Ille-et-Vilaine.
Tout au long de l’année, nous serons encadrés par Bertrand COÜASNON, enseignant-chercheur membre d'IntuiDoc,
Erwan FOUCHÉ, chef de projet chez \href{https://www.soprasteria.com/fr}{Sopra Steria}, Julien BOUVET,
ingénieur chez Sopra Steria également, et Sophie TARDIVEL, responsable et \textit{data scientist}
chez Doptim.

\paragraph{}
Ce projet a pour but de fournir un programme permettant de concevoir des bases d’apprentissage
automatiquement pour l’entraînement de divers systèmes de reconnaissance d’écriture manuscrite
ainsi que leur exploitation. Ces reconnaisseurs seront, par exemple, capables de retranscrire
de manière informatique des documents manuscrits (registres paroissiaux, registres d’état civil,
documents d’entreprise, etc.) pour les rendre plus exploitables. Ce projet permettra donc de gagner
du temps sur la compréhension de documents anciens en rendant l’entraînement de systèmes complexes
plus simple.

\paragraph{}
Dans ce rapport, nous vous détaillerons dans un premier temps le contexte du projet.
Dans un second temps, nous expliciterons les différentes tâches qui doivent être réalisées,
l’objectif mentionné dans le paragraphe précédent restant principal. Nous étudierons également
les différents outils qui sont à notre disposition. Enfin, nous conclurons sur l’organisation
prévisionnelle du projet.