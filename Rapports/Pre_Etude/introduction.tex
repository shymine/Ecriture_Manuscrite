\chapter{Introduction}
\pagenumbering{arabic}
\setcounter{page}{1}

Ce projet nous a été proposé par l’équipe \href{https://www-intuidoc.irisa.fr/}{IntuiDoc}
de l’\href{https://www.irisa.fr/}{IRISA}, en étroite collaboration avec la startup
\href{http://www.doptim.eu}{Doptim} et avec le soutien de Jean-Yves LE CLERC qui dirige
les \href{http://archives.ille-et-vilaine.fr/fr}{archives départementales} d'Ille-et-Vilaine.
Tout au long de l’année, nous serons encadrés par Bertrand COÜASNON, enseignant-chercheur,
Erwan FOUCHE, ingénieur chez \href{https://www.soprasteria.com/fr}{Sopra Steria}, Julien BOUVET,
également ingénieur chez Sopra Steria, et Sophie TARDIVEL, responsable et \textit{data scientist}
chez Doptim.

\paragraph{}
Ce projet a pour but de fournir un programme permettant l’entraînement d’un système
de reconnaissance d’écriture manuscrite et son exploitation. Ce reconnaisseur sera
capable de retranscrire de manière informatique des documents manuscrits
(registres paroissiaux, registres d’état civil, documents d’entreprise, etc.)
de manière automatique. Développé à partir de réseaux de neurones récurrents profonds
(\textit{Deep Learning}), le système de reconnaissance d'écriture manuscrite doit apprendre
par lui-même à retranscrire correctement les textes et a donc besoin d’une base
d’apprentissage. C’est justement cette base que nous devons concevoir.

\paragraph{}
Ce projet ambitieux permettra de gagner du temps sur la compréhension de documents
manuscrits anciens en les rendant plus lisibles une fois informatisés.
Afin de créer cet ensemble d’apprentissage, nous disposons d’une base de données,
la base \href{http://www.maurdor-campaign.org/}{Maurdor}, qui contient des documents
manuscrits libres de droit ainsi que leur retranscription numérique ligne par ligne.

\paragraph{}
Dans ce rapport, nous vous détaillerons dans un premier temps les différentes
tâches qui doivent être réalisées, bien que l’objectif principal soit de pouvoir
entraîner correctement le reconnaisseur. Ensuite, nous expliciterons le contexte
du projet ainsi que les différents outils à notre disposition. Enfin, nous conclurons
sur l’organisation prévisionnelle du projet.