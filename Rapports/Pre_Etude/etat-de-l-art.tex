\chapter{État de l'art}

Dans cette section, nous allons détailler tous les outils à notre disposition ainsi que leur fonctionnement.

\section{Techniques de reconnaissance d’écriture}

La plupart des techniques de reconnaissance de l’écriture sont basées sur des classifieurs
dits dynamiques, c’est-à-dire qu’ils possèdent une mémoire interne ou possèdent une notion
de contexte dans leur analyse. Ces classifieurs permettent un parcours et une division des
données d’entrées et donc d’effectuer une classification pour chacune des divisions repérées. 
		
\subsection{Modèles de Markov Cachés}

Les Modèles de Markov Cachés (MMC) sont des modèles probabilistes basés sur une structure de graphe.
Ils modélisent une séquence basée sur des connaissances \textit{a priori}, comme un langage avec la
forme des lettres. Les MMC sont constitués d’un processus possédant deux composantes stochastiques
(se basant sur le hasard). La première correspond aux états du processus, il s’agit des
observations à chaques instants. La seconde modélise les dépendances temporelles entre les
états non observés, c’est celle-ci qui permet de modéliser les dépendances temporelles markoviennes. 
Les probabilités de passage d’un état à un autre dépendent d’une matrice de transition
et un vecteur de conditions initiales permet de définir l’état de départ. Les MMC permettent de
modéliser la probabilité d’émission d’une observation $y$ (première composante) par un état $x$ qui
est défini par la probabilité $e = p(y|x)$, ce qui donne une seconde matrice représentant
ces probabilités d’émission. Il existe donc deux types d’observation : les observations discrètes,
liées à un alphabet fini, et les observations continues qui prennent leurs valeurs dans un espace de
$\mathbb{R}^N$, avec $N$ la dimension d’un vecteur d’observations. Sur les modèles ainsi générés,
il est possible d’appliquer différents algorithmes suivant les réponses attendues.

\subsection{Champs Aléatoires Conditionnels}

\subsection{Réseaux de Neurones Récurrents}

\section{Détecteur de lignes}

\subsection{Par floutage}

\subsection{Par réseau de neurone à convolution}

\section{Format de description d’image}

\subsection{GEDI}

GEDI est un outil qui permet d’annoter des documents scannés. Il est ainsi très utile pour établir
la vérité terrain. Il met en scène deux types de documents : des images qui correspondent aux documents
scannés ainsi que des documents XML au format GEDI qui permettent de stocker toutes les informations
relatives aux documents scannés. On peut alors avoir, pour chaque document scanné, des informations
relatives à la position des paragraphes, à la langue dans laquelle le texte est écrit, ou encore à
la forme du texte (manuscrit ou imprimé). La vérité terrain que nous aurons au sein de notre projet
aura été établie avec GEDI.

\subsection{PiFF}

Le Pivot File Format (PiFF) est un format de description d’image basé sur JSON et créé par différents
chercheurs français, dont notre encadrant de projet, Bertrand COÜASNON. Ce format est très utile pour l’analyse
de documents puisqu'il permet le partage de jeu de données, le traitement de résultats ainsi que
l’utilisation d’outils déjà existants sans avoir à faire de conversion entre les différents formats
qui pourraient exister. De ce fait, les différentes étapes de l’analyse de documents peuvent être effectuées
par différentes équipes sans qu’il n'y ait de conflit au niveau du format des données, ce qui permet une
collaboration plus facile. Les données présentes en entrée dans le cas de notre projet seront au format PiFF.
Par conséquent, les données après traitement par le logiciel devront également être dans ce format. 

\section{Framework Deep Learning}

\subsection{PyTorch}

\subsection{TensorFlow}

\section{Base de données (pour stocker des images)}