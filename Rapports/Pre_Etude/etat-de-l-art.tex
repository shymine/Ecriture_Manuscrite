\chapter{État de l'art}

Dans cette section, nous allons détailler tous les outils à notre disposition ainsi que leur fonctionnement.

\section{Techniques de reconnaissance d’écriture}

La plupart des techniques de reconnaissance de l’écriture sont basées sur des classifieurs
dits dynamiques, c’est-à-dire qu’ils possèdent une mémoire interne ou possèdent une notion
de contexte dans leur analyse. Ces classifieurs permettent un parcours et une division des
données d’entrées et donc d’effectuer une classification pour chacune des divisions repérées. 
		
\subsection{Modèles de Markov Cachés}

\subsection{Champs Aléatoires Conditionnels}

\subsection{Réseaux de Neurones Récurrents}

\section{Détecteur de lignes}

\subsection{Par floutage}

\subsection{Par réseau de neurone à convolution}

\section{Format de description d’image}

\subsection{GEDI}

GEDI est un outil qui permet d’annoter des documents scannés. Il est ainsi très utile pour établir
la vérité terrain. Il met en scène deux types de documents : des images qui correspondent aux documents
scannés ainsi que des documents XML au format GEDI qui permettent de stocker toutes les informations
relatives aux documents scannés. On peut alors avoir, pour chaque document scanné, des informations
relatives à la position des paragraphes, à la langue dans laquelle le texte est écrit, ou encore à
la forme du texte (manuscrit ou imprimé). La vérité terrain que nous aurons au sein de notre projet
aura été établie avec GEDI.

\subsection{PiFF}

Le Pivot File Format (PiFF) est un format de description d’image basé sur JSON et créé par différents
chercheurs français, dont notre encadrant de projet, Bertrand COÜASNON. Ce format est très utile pour l’analyse
de documents puisqu'il permet le partage de jeu de données, le traitement de résultats ainsi que
l’utilisation d’outils déjà existants sans avoir à faire de conversion entre les différents formats
qui pourraient exister. De ce fait, les différentes étapes de l’analyse de documents peuvent être effectuées
par différentes équipes sans qu’il n'y ait de conflit au niveau du format des données, ce qui permet une
collaboration plus facile. Les données présentes en entrée dans le cas de notre projet seront au format PiFF.
Par conséquent, les données après traitement par le logiciel devront également être dans ce format.