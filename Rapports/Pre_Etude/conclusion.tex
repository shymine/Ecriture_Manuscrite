\chapter{Conclusion}

Dans le cadre de notre 4\up{ème} année en informatique, nous avons pour mission de
mener à bien un projet de manière plus précise et plus structurée qu’en 3\up{ème} année.
Notre projet, en partenariat avec les archives départementales d’Ille-et-Vilaine,
l’équipe IntuiDoc et Doptim, consiste à créer un logiciel qui aidera les chercheurs et
ingénieurs à faire progresser la reconnaissance d’écriture manuscrite, afin de rendre
des documents anciens plus accessibles au grand public et plus compréhensibles. L’objectif
de ce logiciel est de générer des données d’apprentissage pour un système de reconnaissance
d’écriture manuscrite.

\paragraph{}
Notre groupe se compose de 8 étudiants : Enzo CRANCE, Kevin DESPOULAINS, Timothée NEITTHOFFER,
Laure DU MESNILDOT, Charlotte RICHARD, Valentin FOUCHER, Gaël GENDRON et Corentin GUILLOUX.
Trois d’entre nous partiront en mobilité internationale au second semestre : Kevin DESPOULAINS,
Gaël GENDRON et Corentin GUILLOUX. Il nous faudra ainsi prendre en compte dans la réalisation
de notre projet ce changement d’effectif. L’étude du projet, la répartition des tâches et la
planification ont donc été faites bien en amont des phases de développement pour permettre à
ceux qui seront encore présents au second semestre de ne pas prendre de retard.

\paragraph{}
Nous avons dans un premier temps étudié ce qui nous était demandé de faire avant de
décider des technologies que nous allions utiliser. Nous avons bien sûr commencé à
étudier les différents outils (langages, API, etc.) que nous pourrions utiliser et qui
feront l’objet du prochain rapport.

\paragraph{}
Nous avons ensuite rédigé notre cahier des charges en reprenant les différentes fonctionnalités
que nous souhaitons développer, en accord avec Bertrand COÜASNON, tout en prenant en compte
les éléments extérieurs avec lesquels notre programme doit communiquer.

\paragraph{}
Nous avons également distribué les rôles qui nous semblent importants pour veiller au bon
déroulement du projet sans mettre trop de pression sur les personnes les endossant.
Un planning prévisionnel a également été rédigé.