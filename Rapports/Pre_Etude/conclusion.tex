\chapter{Conclusion}

Dans le cadre de notre 4ème année en informatique, nous avons pour mission de
mener à bien un projet de manière plus précise et plus structurée qu’en 3ème année.
Notre projet, en partenariat avec les archives départementales d’Ille-et-Vilaine,
l’équipe IntuiDoc et Doptim, consiste à faire progresser la reconnaissance d’écriture
manuscrite afin de rendre plus accessibles et plus compréhensibles des documents anciens.

\paragraph{}
Notre groupe se compose de 8 étudiants dont 3 qui partiront en mobilité internationale
au second semestre, ce que nous devons prendre en compte dans la réalisation de notre
projet. L’étude du projet, la répartition des tâches et la planification ont donc été
fait bien en amont des phases de développement pour permettre à ceux qui seront encore
présent au second semestre de ne pas prendre de retard.

\paragraph{}
Nous avons dans un premier temps étudié ce qui nous était demandé de faire avant de
décider des technologies que nous allions utiliser. Nous avons bien sûr commencé à
étudier les différents outils (langages, API, etc.) que nous pourrions utiliser mais
cela fera l’objet du prochain rapport.

\paragraph{}
Nous avons ensuite rédigé notre cahier des charges reprenant les différentes fonctionnalités
que nous souhaitons développer, en accord avec Bertrand COÜASNON, tout en prenant en compte
les éléments extérieurs avec lesquels notre programme doit communiquer.

\paragraph{}
Nous avons également distribué les rôles qui nous semblent important pour veiller au bon
déroulement du projet sans mettre trop de pression sur les personnes les endossant.
Un planning prévisionnel a également été rédigé.